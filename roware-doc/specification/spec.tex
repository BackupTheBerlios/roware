%------------------------------------------------------------------------------
% Row@re . Spec
%------------------------------------------------------------------------------
\newcommand{\roware}{\textbf{ROW@RE}\ }

\newcommand{\usecase}[5]{
\begin{minipage}{\textwidth}
\textbf{Vorbedingung}\\
#1 \\
\textbf{Beschreibung}\\
#2 \\
\textbf{Nachbedingung Erfolg}\\
#3 \\
\textbf{Nachbedingung Fehlschlag}\\
#4 \\
\textbf{Interaktionen}\\
#5
\end{minipage}
}

\documentclass[11pt, a4paper]{scrbook}
\usepackage{ngerman}
\usepackage[T1]{fontenc}
\usepackage[latin1]{inputenc}
\usepackage{graphicx,color}
\usepackage{eso-pic}
\usepackage{ae}

\definecolor{lightgray}{gray}{.75}
\renewcommand{\familydefault}{\sfdefault}

\begin{document}

  \AddToShipoutPicture{%
    \AtTextCenter{%
      \makebox(0,0)[c]{\resizebox{\textwidth}{!}{%
        \rotatebox{45}{\textsf{\textbf{\color{lightgray}DRAFT}}}}} 
    }
  }


\begin{titlepage}
\renewcommand{\thefootnote}{\fnsymbol{footnote}}
{\Huge
\raggedright
\roware \\
\huge Integriertes EDV-System zur Abwicklung von Ruderegatten
\rule{\textwidth}{0.75pt}
\par
}
\begin{flushleft}
\normalsize
Version 0.1\\[2ex]
\today \\
Oliver Urschel\footnote{Aachen/Bad Kreuznach, \texttt{<urschcom@users.berlios.de>}}\\
Tammo van Lessen\footnote{Stuttgart/Mainz, \texttt{<vanto@users.berlios.de>}}\\
\end{flushleft}

\vfill

{\parindent=0cm
\Huge Spezifikation
}


\setcounter{footnote}{0}
\end{titlepage}

\onecolumn
{\parindent=0cm\thispagestyle{empty}
\copyright{} Copyright 2003 Tammo van Lessen
\bigskip

Die Verteilung dieses Dokuments in elektronischer oder gedruckter
Form ist gestattet, solange sein Inhalt einschlie"slich Autoren- und 
Copyright-Angabe unver"andert bleibt und die Verteilung kostenlos
erfolgt, abgesehen von einer Ge\-b"uhr f"ur den Datentr"ager, den
Kopiervorgang usw.
\bigskip

Die in dieser Publikation erw"ahnten Software- und Hardware-Bezeichnungen sind
in den meisten F"allen auch eingetragene Warenzeichen und unterliegen als
solche den gesetzlichen Bestimmungen.
\bigskip

\vfill

Dieses Dokument wurde mit \LaTeX{} gesetzt.
}
\clearpage
\newpage


\thispagestyle{empty}
%\twocolumn
%-------------------------------------------------------------------------
\chapter{Einleitung}
\section{Zielsetzung}
In diesem Dokument sind die Anforderungen an das Projekt \roware
detailliert spezifiziert. Das Produkt wird streng nach den hier aufgef�hrten 
Anforderungen entwickelt. Der Entwicklungsprozess gilt als erfolgreich 
abgeschlossen, wenn das Produkt allen hier aufgef�hrten Aspekten gerecht wird.
�nderungen an diesem Dokument sind nach Abschluss der Spezifikationsphase nicht mehr ohne das Einverst�ndnis des \roware-Teams m�glich.

\section{Produktziele}
Bei \roware handelt es sich um ein Programmsystem, das alle Phasen einer 
Regatta unterst�tzt. Dazu geh�rt im Vorfeld der Regatta die Erfassung der eingegangenen Meldungen, die Durchf�hrung der Startverlosung, das Erstellen des Meldeergebnisses. W�hrend der Regatta unterst�tzt es das Regattab�ro, Waage und Kasse.
Nach der Regatta k�nnen automatisch korrigierte Meldeergebnisse erstellt werden.

\section{Produktumgebung}
Das Programm soll auf beliebigen Plattformen lauff�hig sein. Deshalb wird es als Webanwendung in PHP implementiert. Der Benutzer ben�tig nur einen modernen Browser mit JavaScript-Unterst�tzung.

\section{Produktfunktionen}
Das Produkt stellt dem Regattaveranstalter folgende Funktionen zur Verf�gung

\textbf{Ausschreibung}
\begin{itemize}
\item Ausschreibung erstellen
\item Ausschreibung ver�ffentlichen (HTML, PDF, xregatta)
\end{itemize}

\textbf{Meldeschluss}
\begin{itemize}
\item Daten einlesen aus xregatta-format
\item Daten eingeben (Papiermeldung)
\item Daten �berpr�fen und korrigieren
\item Rennen verlosen
\item Rennzeiten festlegen
\item Meldeergebnis ver�ffentlichen (HTML, PDF, xregatta)
\end{itemize}

\textbf{Regattab�ro}
\begin{itemize}
\item Nachmeldung/Abmeldung/Ummeldung erfassen und damit verbundene Rennverschiebungen verarbeiten
\item Jugendlizenzen kontrollieren/erfassen
\item Schiedsrichtereinsatzplan erstellen, ver�ffentlichen
\item Waage erfassen
\item Kasse Buchungen, Quittung erstellen, Berichte erstellen
\item Rennen verlegen
\item Einspr�che erfassen
\item Schiedsrichterbericht erstellen
\item Urkunden drucken
\item Ergebnisse ver�ffentlichen (HTML, PDF, xregatta)
\end{itemize}

\textbf{Start/Ziel}
\begin{itemize}
\item Startberechtigung kontrollieren (�bergewicht, Zusatzgewicht, unerlaubte Renngemeinschaften usw.)
\item Ausschluss von Mannschaften erfassen (Geldstrafen???)
\item Mannschaften zu Rennen einteilen
\item Rennen starten
\item Zwischenzeiten nehmen
\item Unf�lle erfassen (Kentern, Bootsschaden usw.)
\item Einspr�che erfassen
\item Zeit nehmen (Ziel)
\item Zielfotos hochladen
\end{itemize}

\chapter{Anforderungen}
Die Anforderungen werden in Form von Anwendungsf�llen (Use-Cases) erkl�rt.
Sie beschreiben die Funktionalit�t des zu erstellenden Systems.

\section{Aktoren}
\subsection{Benutzer}
Vor der Anmeldung an das System sind haben alle Benutzer die gleiche Funktion und die gleichen M�glichkeiten. Nach der Anmeldung ergibt sich durch das Rechtemanagment eine neue Situation, die Benutzer werden in weitere Gruppen mit unterschiedlichen Rechten aufgeteilt.
\begin{itemize}
\item Benutzer im Regattab�ro (\emph{ACT-Admin})
\item Benutzer an der Wage (\emph{ACT-Waage})
\item Benutzer am Start (\emph{ACT-Start})
\item Benutzer im Ziel (\emph{ACT-Ziel})
\item Benutzer an der Strecke (\emph{ACT-Zeit})
\item Gast (\emph{ACT-Gast})
\item TODO: weitere Benutzer?
\end{itemize}

\section{Use-Cases}
Die hier beschriebenen Gesch�ftsprozesse beschreiben ausschlie�lich die Interaktion zwischen Benutzer und Maschine. Implementations- und Gestaltungsdetails sind hier falsch.

\subsection{Am System anmelden}
\usecase{Browser ist gestartet und zeigt auf http://ip-adresse/}
	{Der Benutzer tr�gt seinen Namen und sein Passwort in Textfelder ein.
	Danach best�tigt er 
	mit \emph{Return} oder klickt auf \emph{OK}. Darauf hin wird das 
	Passwort �berpr�ft.}
	{Der Benutzer ist nun angemeldet, es erscheint das Hauptfenster}
	{Es wird die Fehlermeldung ``Falsches Kennwort'' angezeigt, das 
	Anmeldefenster bleibt angezeigt.}
	{Benutzer}

\subsection {Ausschreibung erstellen}
\usecase{}{TODO}{}{}{ACT-Admin}

\subsection {Ausschreibung ver�ffentlichen (HTML, PDF, xregatta)}
\usecase{}{TODO}{}{}{ACT-Admin}

\subsection {Daten einlesen aus xregatta-format}
\usecase{}{TODO}{}{}{ACT-Admin}

\subsection {Daten eingeben (Papiermeldung)}
\usecase{}{TODO}{}{}{ACT-Admin}

\subsection {Daten �berpr�fen und korrigieren}
\usecase{}{TODO}{}{}{ACT-Admin}

\subsection {Rennen verlosen}
\usecase{}{TODO}{}{}{ACT-Admin}

\subsection {Rennzeiten festlegen}
\usecase{}{TODO}{}{}{ACT-Admin}

\subsection {Meldeergebnis ver�ffentlichen (HTML, PDF, xregatta)}
\usecase{}{TODO}{}{}{ACT-Admin}

\subsection {Nachmeldung/Abmeldung/Ummeldung erfassen und damit verbundene Rennverschiebungen verarbeiten}
\usecase{}{TODO}{}{}{ACT-Admin}

\subsection {Jugendlizenzen kontrollieren/erfassen}
\usecase{}{TODO}{}{}{ACT-Admin}

\subsection {Schiedsrichtereinsatzplan erstellen, ver�ffentlichen}
\usecase{}{TODO}{}{}{ACT-Admin}

\subsection {Waage erfassen}
\usecase{}{TODO}{}{}{ACT-Waage, ACT-Admin}

\subsection {Kasse Buchungen, Quittung erstellen, Berichte erstellen}
\usecase{}{TODO}{}{}{ACT-Admin}

\subsection {Rennen verlegen}
\usecase{}{TODO}{}{}{ACT-Admin}

\subsection {Einspr�che erfassen}
\usecase{}{TODO}{}{}{ACT-Admin}

\subsection {Schiedsrichterbericht erstellen}
\usecase{}{TODO}{}{}{ACT-Admin}

\subsection {Urkunden drucken}
\usecase{}{TODO}{}{}{ACT-Admin}

\subsection {Ergebnisse ver�ffentlichen (HTML, PDF, xregatta)}
\usecase{}{TODO}{}{}{ACT-Admin}

\subsection {Startberechtigung kontrollieren (�bergewicht, Zusatzgewicht, unerlaubte Renngemeinschaften usw.)}
\usecase{}{TODO}{}{}{ACT-Start, ACT-Admin}

\subsection {Ausschluss von Mannschaften erfassen (Geldstrafen???)}
\usecase{}{TODO}{}{}{ACT-Start, ACT-Ziel}

\subsection {Mannschaften zu Rennen einteilen}
\usecase{}{TODO}{}{}{ACT-Start, ACT-Admin}

\subsection {Rennen starten}
\usecase{}{TODO}{}{}{ACT-Start, ACT-Admin}

\subsection {Zwischenzeiten nehmen}
\usecase{}{TODO}{}{}{ACT-Zeit, ACT-Admin}

\subsection {Unf�lle erfassen (Kentern, Bootsschaden usw.)}
\usecase{}{TODO}{}{}{ACT-Zeit, ACT-Start, ACT-Ziel, ACT-Admin}

\subsection {Einspr�che erfassen}
\usecase{}{TODO}{}{}{ACT-Admin}

\subsection {Zeit nehmen (Ziel)}
\usecase{}{TODO}{}{}{ACT-Ziel, ACT-Admin}

\subsection {Zielfotos hochladen}
\usecase{}{TODO}{}{}{ACT-Ziel, ACT-Admin}


\chapter{Graphische Benutzeroberfl�che}
TODO
\section{Anmeldefenster}
TODO
\section{Hauptfenser}
TODO
\chapter{Daten}
TODO Datenstruktur beschreiben
%-------------------------------------------------------------------------
\chapter{Offene Fragen}
%\begin{itemize}
%\item �ber welchen Zeitraum werden die Daten gespeichert
%\end{itemize}
\end{document}
